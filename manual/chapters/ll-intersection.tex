\chapter{两直线的交点}

求解两直线交点的方程\cite{LINE}:

\begin{align*}
  \begin{vmatrix}
    x & y & 1\\
    x_1 & y_1 & 1\\
    x_2 & y_2 & 1
  \end{vmatrix} & = 0\\
  \begin{vmatrix}
    x & y & 1\\
    x_3 & y_3 & 1\\
    x_4 & y_4 & 1
  \end{vmatrix} & = 0
\end{align*}

注意, 两个方程的系数都是行列式, 解得:

\begin{align*}
  x &= \cfrac{
      \begin{vmatrix}
        \begin{vmatrix} x_1 & y_1 \\ x_2 & y_2 \end{vmatrix} & \begin{vmatrix} x_1 & 1 \\ x_2 & 1 \end{vmatrix} \\
        \begin{vmatrix} x_3 & y_3 \\ x_4 & y_4 \end{vmatrix} & \begin{vmatrix} x_3 & 1 \\ x_4 & 1 \end{vmatrix} 
      \end{vmatrix}}{
      \begin{vmatrix}
        \begin{vmatrix} x_1 & 1 \\ x_2 & 1 \end{vmatrix} & \begin{vmatrix} y_1 & 1 \\ y_2 & 1 \end{vmatrix} \\
        \begin{vmatrix} x_3 & 1 \\ x_4 & 1 \end{vmatrix} & \begin{vmatrix} y_3 & 1 \\ y_4 & 1 \end{vmatrix} 
      \end{vmatrix}} 
    = \cfrac{
      \begin{vmatrix}
        \begin{vmatrix} x_1 & y_1 \\ x_2 & y_2 \end{vmatrix} & x_1 - x_2 \\
        \begin{vmatrix} x_3 & y_3 \\ x_4 & y_4 \end{vmatrix} & x_3 - x_4 
      \end{vmatrix}}{
      \begin{vmatrix}
        x_1 - x_2 & y_1 - y_2 \\
        x_3 - x_4 & y_3 - y_4 
      \end{vmatrix}}\\
  y &= \cfrac{
      \begin{vmatrix}
        \begin{vmatrix} x_1 & y_1 \\ x_2 & y_2 \end{vmatrix} & \begin{vmatrix} y_1 & 1 \\ y_2 & 1 \end{vmatrix} \\
        \begin{vmatrix} x_3 & y_3 \\ x_4 & y_4 \end{vmatrix} & \begin{vmatrix} y_3 & 1 \\ y_4 & 1 \end{vmatrix} 
      \end{vmatrix}}{
      \begin{vmatrix}
        \begin{vmatrix} x_1 & 1 \\ x_2 & 1 \end{vmatrix} & \begin{vmatrix} y_1 & 1 \\ y_2 & 1 \end{vmatrix} \\
        \begin{vmatrix} x_3 & 1 \\ x_4 & 1 \end{vmatrix} & \begin{vmatrix} y_3 & 1 \\ y_4 & 1 \end{vmatrix} 
      \end{vmatrix}}
    = \cfrac{
      \begin{vmatrix}
        \begin{vmatrix} x_1 & y_1 \\ x_2 & y_2 \end{vmatrix} & y_1 - y_2 \\
        \begin{vmatrix} x_3 & y_3 \\ x_4 & y_4 \end{vmatrix} & y_3 - y_4 
      \end{vmatrix}}{
      \begin{vmatrix}
        x_1 - x_2 & y_1 - y_2 \\
        x_3 - x_4 & y_3 - y_4 
      \end{vmatrix}}
\end{align*}

进一步化简得到\footnote{\url{https://en.wikipedia.org/wiki/Line–line_intersection}}:

\begin{align*}
  x &= \cfrac{(x_1y_2-y_1x_2)(x_3-x_4)-(x_1-x_2)(x_3y_4-y_3x_4)}{(x_1-x_2)(y_3-y_4)-(y_1-y_2)(x_3-x_4)}\\
  y &= \cfrac{(x_1y_2-y_1x_2)(y_3-y_4)-(y_1-y_2)(x_3y_4-y_3x_4)}{(x_1-x_2)(y_3-y_4)-(y_1-y_2)(x_3-x_4)}
\end{align*}

上述方法给出的交点坐标公式在 TikZ 环境中的计算稳定性不够好, 经常出现 {\ttfamily Dimension too large} 错误, 
究其原因是分母可能有时会比较小. 下面给出一个计算更稳定的公式.

我们可以给出两条直线的参数方程:

\begin{align*}
  \intertext{直线 $L_1$ 的方程:}
  \begin{bmatrix} x\\ y \end{bmatrix} &= 
    \begin{bmatrix}x_1\\ y_1\end{bmatrix}
    + s\begin{bmatrix}x_2-x_1\\y_2-y_1\end{bmatrix}\\
  \intertext{直线 $L_2$ 的方程:}
  \begin{bmatrix} x\\ y \end{bmatrix} &= 
    \begin{bmatrix}x_3\\ y_3\end{bmatrix}
    + t\begin{bmatrix}x_4-x_3\\ y_4-y_3\end{bmatrix}
\end{align*}

可以解出 $s,t$:

\begin{align*}
  s &= \cfrac{
    \begin{vmatrix}x_1-x_3 & x_3-x_4\\y_1-y_3 & y_3-y_4 \end{vmatrix}
  }{\begin{vmatrix}x_1-x_2 & x_3-x_4\\y_1-y_2 & y_3-y_4 \end{vmatrix}}\\
  t &= \cfrac{
    \begin{vmatrix}x_1-x_3 & x_1-x_2\\y_1-y_3 & y_1-y_2 \end{vmatrix}
  }{\begin{vmatrix}x_1-x_2 & x_3-x_4\\y_1-y_2 & y_3-y_4 \end{vmatrix}}
\end{align*}

我们也可从几何的角度来分析:

\begin{figure}[H]
\centering
\begin{tikzpicture}
  \coordinate (A) at (0,0);
  \coordinate (B) at (4,2);
  \coordinate (C) at (0,2);
  \coordinate (D) at (3,0);
  \coordinate [intersect={A,B,C,D}] (E); 
  \coordinate [translate={C,D,A}] (F);
  \foreach \p in {A,B,C,D,E,F}
    \fill[orange] (\p) circle (2pt);
  \draw[thick,-latex] (A) -- (B);
  \draw[thick,-latex] (C) -- (D);
  \draw[thick,-latex] (A) -- (F);
  \draw[thick,-latex] (A) -- (C);
  \fill[yellow,opacity=0.25] (A) -- (E) -- (F) -- cycle;
  \fill[green,opacity=.25] (A) -- (C) -- (F) -- cycle;
  \draw[dashed] (B) -- (F);
  \draw (A) node[left] {$A$};
  \draw (B) node[right] {$B$};
  \draw (C) node[above left] {$C$};
  \draw (D) node[below right] {$D$};
  \draw (E) node[above] {$E$};
  \draw (F) node[below left] {$F$};
\end{tikzpicture}
\end{figure}

\begin{align*}
  \overrightarrow{AE} &= s \overrightarrow{AB}\\
  s &= \cfrac{S_{\triangle{AEF}}}{S_{\triangle{ABF}}}\\
    &= \cfrac{S_{\triangle{ACF}}}{S_{\triangle{ABF}}}\\
    &= \cfrac{\overrightarrow{AF}\times\overrightarrow{AC}}{\overrightarrow{AF}\times\overrightarrow{AB}}\\
    &= \cfrac{\overrightarrow{CD}\times\overrightarrow{AC}}{\overrightarrow{CD}\times\overrightarrow{AB}}
\end{align*}

为了保证数值计算的稳定性, 可以对下面的方程进行列主元消元法求解:
\begin{align*}
  x_1+s(x_2-x_1) &= x3+t(x_4-x3)\\
  y_1+s(y_2-y_1) &= y3+t(y_4-y3)
\end{align*} 