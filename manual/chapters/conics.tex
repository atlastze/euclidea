\chapter{圆锥曲线 Conics}

尽管 tikz 内置的 \verbum{ellipse} 和 \verbum{parabola} 绘制椭圆和抛物线,
这里定义了 \mintinline{latex}|\ellipse| 和 \mintinline{latex}|\parabola|.

% ---------------------------------
\section{椭圆 Ellipse}

\emph{调用方式}

\begin{tcolorbox}{}
\mint{latex}{\ellipse [options] (a,b)}
\end{tcolorbox}

\emph{参数说明}

\begin{description}
  \item[a, b] 半长轴长 (semi-major axis) 和半短轴长 (semi-minor axis), 默认单位为 cm, 可指定单位, 如 \verbum{(4cm, 3cm)}
\end{description}

返回中心为原点的椭圆曲线: $\dfrac{x^2}{a^2}+\dfrac{y^2}{b^2}=1$.

\begin{remark*}
当指定椭圆曲线 (ellipse) 的 \verbum{domain} (default: \verbum{domain=-180:180})时, \verbum{domain} 是下列参数方程中 $t$ 的取值范围:
\begin{align*}
  \begin{cases}
  x = a \cos t,\\y = b \sin t
  \end{cases}
\end{align*}
\end{remark*}

\emph{示例}

使用 tikz 内置曲线:

\showcode{snippets/conics/ellipse1.tex}

使用 \mintinline{latex}|\ellipse| 命令:

\showcode{snippets/conics/ellipse2.tex}

% ---------------------------------
\section{抛物线 Parabola}

\emph{调用方式}

\begin{tcolorbox}{}
\mint{latex}{\parabola [options] (p)}
\end{tcolorbox}

\emph{参数说明}

\begin{description}
  \item[p] 焦准距或焦参数 (focal parameter), 默认单位为 cm, 可指定单位, 如 \verbum{1cm}
\end{description}

返回中心为原点的抛物线: $x^2 = 2py$.

\begin{remark*}
抛物线的焦点到顶点的距离为$\dfrac{p}{2}$, 抛物线的准线到顶点的距离也是$\dfrac{p}{2}$, 抛物线的半正焦弦或半通径 (semi-latus rectum)也为 $p$(参考附录\ref{ch:latus-rectum}).

当指定绘制抛物线 (parabola) 的 \verbum{domain} (default: \verbum{domain=-2:2})时, \verbum{domain} 是下列参数方程中 $t$ 的取值范围:
\begin{align*}
  \begin{cases}
  x = 2pt,\\y = 2pt^2
  \end{cases}
\end{align*}
\end{remark*}

\emph{示例}

使用 tikz 内置曲线:

\showcode{snippets/conics/parabola1.tex}

使用 \mintinline{latex}|\parabola| 命令:

\showcode{snippets/conics/parabola2.tex}

% ---------------------------------
\section{双曲线 Hyperbola 与渐近线 Asymptote}

\emph{调用方式}

\begin{tcolorbox}{}
\mint{latex}{\hyperbola [options] (a,b);}
\end{tcolorbox}

或

\begin{tcolorbox}{}
\mint{latex}{\asymptote [options] (a,b);}
\end{tcolorbox}

\emph{参数说明}

\begin{description}
  \item[a, b] 半实轴长 (semi-major axis) 和半虚轴长 (semi-minor axis), 默认单位为 cm, 可指定单位, 如 \verbum{(4cm, 3cm)}
\end{description}

分别返回中心在原点的双曲线: $\dfrac{x^2}{a^2}-\dfrac{y^2}{b^2}=1$ 和渐近线: $y = \pm \dfrac{b}{a}x$.

\emph{示例}

\showcode{snippets/conics/hyperbola1.tex}

\begin{remark*}
当指定绘制双曲线 (hyperbola) 的 \verbum{domain} (default: \verbum{domain=-1.5:1.5})时, \verbum{domain} 是下列双曲线参数方程中 $t$ 的取值范围:
\begin{align*}
  \begin{cases}
  x = \cosh t,\\y = \sinh t
  \end{cases}
\end{align*}

$t$ 的几何意义: 射线出原点交单位双曲线 $x^2-y^2=1$ 于 $(\cosh t, y = \sinh t)$,
$t$ 是射线,双曲线和 $x$ 轴围成的面积的二倍. 对于双曲线上位于 $x$ 轴下方的点, 这个面积被认为是负值.

当指定绘制渐进线 (asymptote) 的 \verbum{domain} (default: \verbum{domain=-2:2})时, \verbum{domain} 是下列直线方程中 $x$ 的取值范围:
\begin{align*}
  y = \pm \dfrac{b}{a} x
\end{align*}
\end{remark*}

\emph{示例}

\showcode{snippets/conics/hyperbola2.tex}