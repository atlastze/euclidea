\ProvidesFile{tikzlibraryeuclidea.code.tex}[2024/02/22 v1.3.2 A tikz library for plane geometry]

\usetikzlibrary{math,calc,quotes}

% https://tex.stackexchange.com/questions/455991/pgfmath-function-for-strings-and-numbers
% Solving the error:
% Package PGF Math: Could not parse input 'A' as a floating 
% point number, sorry. The unreadable part was near 'A'..
\pgfkeys{
  /pgf/fpu/handlers/invalid number/.code = {%
    \pgfmathfloatparsenumber{3Y0.0e0]}%
  }
}

\makeatletter

% 注意: 计算过程是保留坐标单位(pt)的, 所以存在乘除法单位的问题, 首先数值始终携带单位, 
% 在 calc 运算时有的需要转换为标量; 将坐标转换为 pt 值, 数值可能超出限值, 出现 
% Dimension too large 错误, 在计算长度时及时进行缩小 
% https://tex.stackexchange.com/questions/475556/tikz-why-is-dimension-too-large
% 具体方法是修改默认的 1cm, 如: [scale=1.0,x=0.5cm,y=0.5cm]
% 注意此处的变量不要和 tikzpicture 环境重名, 否则被替换掉
% triangle centers: 
% https://mathworld.wolfram.com/BarycentricCoordinates.html
\tikzmath{
  % 采用列主元消元法求直线 P1P2 与直线 P3P4 的交点 P 位置参数 s: s = P1P/P1P2
  function intersectll(\x1,\y1,\x2,\y2,\x3,\y3,\x4,\y4) {
    \a1 = \x2-\x1; \b1 = \x3-\x4; \c1 = \x3-\x1;
    \a2 = \y2-\y1; \b2 = \y3-\y4; \c2 = \y3-\y1;
    \dmax = max(max(abs(\a1),abs(\a2)), max(abs(\b1),abs(\b2)));
    \a1 = \a1/\dmax; \b1 = \b1/\dmax; \c1=\c1/\dmax;
    \a2 = \a2/\dmax; \b2 = \b2/\dmax; \c2=\c2/\dmax;
    if abs(\a1) < abs(\a2) then {
      \temp = \a1; \a1 = \a2; \a2 = \temp;
      \temp = \b1; \b1 = \b2; \b2 = \temp;
      \temp = \c1; \c1 = \c2; \c2 = \temp;
    };
    \b1 = \b1/\a1; \c1 = \c1/\a1; \a1 = 1.0;
    \b2 = \b2-\a2*\b1; \c2 = \c2-\a2*\c1; \a2 = 0.0;
    \n2 = \c2/\b2; \n1 = \c1-\b1*\n2;
    return \n1;
  };
}

\tikzset{
  % specifying start and end with modifiers(see tikz manual 13.5) 
  % commands supporting partway modifiers:
  % radical axis, perpendicular bisector, perpendicular, parallel
  start modifier/.initial = 0,
  start modifier/.default = 0,
  end modifier/.initial = 1,
  end modifier/.default = 1,
  % ========== Coordinates Transformations ==========
  % affine={A,B,k}: returns affine combination of two points
  % with affine ratio, i.e. A + k * ( B - A )
  affine/.style args = {#1,#2,#3}{
    insert path = {
      ($(#1)!{#3}!(#2)$)
    }
  },
  % midpoint={A,B}: returns midpoint of AB.
  midpoint/.style args = {#1,#2}{
    insert path = {
      ($(#1)!.5!(#2)$)
    }
  },
  % translate={A,B,C}: returns translation of C by 
  % the vector AB, i.e. C + ( B - A )
  translate/.style args = {#1,#2,#3}{
    insert path = {
      ($(#3)+(#2)-(#1)$)
    }
  },
  % reflect={A,B,C}: reflects point C across line AB. 
  reflect/.style args = {#1,#2,#3}{
    insert path = {
      let
        \p{ft} = ($(#1)!(#3)!(#2)$),% perpendicular foot
      in ($(#3)!2!(\p{ft})$)
    }
  },
  % project={A,B,C}: projects point C onto line AB. 
  project/.style args = {#1,#2,#3}{
    insert path = {
      ($(#1)!(#3)!(#2)$)
    }
  },
  % inverse={O,A,P}: returns inverse point P with respect to 
  % a reference circle(O,A). 
  inverse/.style args = {#1,#2,#3}{
    insert path = {
      let
        \p{OA} = ($(#2)-(#1)$),
        \p{OP} = ($(#3)-(#1)$),
        \n{r} = {veclen(\p{OA})},
        \n{d} = {veclen(\p{OP})},
        \n1 = {scalar((\n{r}/\n{d}))},
      in ($(#1)!\n1*\n1!(#3)$)
    }
  },
  revolve/scale/.initial = 1,% angle scale
  revolve/@angle/.initial = 90,
  revolve/@argn/.initial = 0,% arguments count
  % set revolve/@angle with certain degrees or angle of a vector
  revolve/@set angle 1/.code args = {#1}{
    \pgfmathanglebetweenpoints
      {\pgfpoint{0cm}{0cm}}
      {\pgfpointanchor{#1}{center}}
    \pgfkeysalso{/tikz/revolve/@angle = \pgfmathresult}
    \typeout{=========================}
    \typeout{/tikz/revolve/@angle:\pgfkeysvalueof{/tikz/revolve/@angle}}
    \typeout{=========================}
  },
  % set revolve/@angle with angle between two position vectors
  revolve/@set angle 2/.code args = {#1,#2}{
    \pgfmathanglebetweenpoints
      {\pgfpointanchor{#1}{center}}
      {\pgfpointanchor{#2}{center}}
    \pgfkeysalso{/tikz/revolve/@angle = \pgfmathresult}
    \typeout{=========================}
    \typeout{/tikz/revolve/@angle:\pgfkeysvalueof{/tikz/revolve/@angle}}
    \typeout{=========================}
  },
  % set revolve/@angle with angle {A,B,C}, angle between two sides
  % (A is apex, B is the start point, C is the end point) 
  revolve/@set angle 3/.code args = {#1,#2,#3}{
    \pgfmathanglebetweenlines
      {\pgfpointanchor{#1}{center}}
      {\pgfpointanchor{#2}{center}}
      {\pgfpointanchor{#1}{center}}
      {\pgfpointanchor{#3}{center}}
    \pgfkeysalso{/tikz/revolve/@angle = \pgfmathresult}
    \typeout{=========================}
    \typeout{/tikz/revolve/@angle:\pgfkeysvalueof{/tikz/revolve/@angle}}
    \typeout{=========================}
  },
  % set revolve/@angle with angle between two vectors(ccw, AB and CD) 
  revolve/@set angle 4/.code args = {#1,#2,#3,#4}{
    \pgfmathanglebetweenlines
      {\pgfpointanchor{#1}{center}}
      {\pgfpointanchor{#2}{center}}
      {\pgfpointanchor{#3}{center}}
      {\pgfpointanchor{#4}{center}}
    \pgfkeysalso{/tikz/revolve/@angle = \pgfmathresult}
    \typeout{=========================}
    \typeout{/tikz/revolve/@angle:\pgfkeysvalueof{/tikz/revolve/@angle}}
    \typeout{=========================}
  },
  revolve/angle/.code = {%
    \pgfmathfloatparsenumber{#1}
    \pgfmathfloattomacro{\pgfmathresult}{\F}{\M}{\E}
    \ifnum \F < 3%number
      \pgfmathparse{#1}
    \else
      % Compute the Number of Arguments
      \pgfutil@for\arg:=#1\do{
        \pgfmathparse{int(add(\pgfkeysvalueof{/tikz/revolve/@argn},1))}
        \pgfkeysalso{/tikz/revolve/@argn = \pgfmathresult}
      }
      \ifnum \pgfkeysvalueof{/tikz/revolve/@argn} = 1
        \tikzset{revolve/@set angle 1 = {#1}}
      \else\ifnum \pgfkeysvalueof{/tikz/revolve/@argn} = 2
        \tikzset{revolve/@set angle 2 = {#1}}
      \else\ifnum \pgfkeysvalueof{/tikz/revolve/@argn} = 3
        \tikzset{revolve/@set angle 3 = {#1}}
      \else\ifnum \pgfkeysvalueof{/tikz/revolve/@argn} = 4
        \tikzset{revolve/@set angle 4 = {#1}}
      \else
        \pgferror{"Incorrect number of arguments!"}
      \fi\fi\fi
      \fi
    \fi 
    \pgfkeysalso{/tikz/revolve/@angle = \pgfmathresult}
  },
  % revolve={A,B}: rotates point B by the angle around point A.
  revolve/.style args = {#1,#2}{
    insert path = {
      let 
        \n1 = {\pgfkeysvalueof{/tikz/revolve/@angle}},
        \n2 = {\pgfkeysvalueof{/tikz/revolve/scale}}
      in ($(#1)!1!\n1*\n2:(#2)$)
    }
  },
  % angle bisector={A,B,C}: alias for [revolve/angle={A,B,C},revolve/scale=.5,revolve={A,B}]
  angle bisector/.style args = {#1,#2,#3}{
    revolve/angle={#1,#2,#3},revolve/scale=.5,revolve={#1,#2}
  },
  % erect={A,B}: alias for [revolve/angle=90,revolve={A,B}]
  erect/.style args = {#1,#2}{
    revolve/angle=90,revolve={#1,#2}
  },
  % equilateral={A,B}: alias for [revolve/angle=60,revolve={A,B}]
  equilateral/.style args = {#1,#2}{
    revolve/angle=60,revolve={#1,#2}
  },
  % cut a line segment of a certain length on a straight line
  intercept/@length/.initial = 1cm,
  intercept/scale/.initial = 1,% length scale
  intercept/length/.code = {% set length by distance of segment
    \pgfutil@in@{,}{#1}
    \ifpgfutil@in@%compute segment length
      \euclidea@ComputeLength#1\euclidea@stop
      \pgfkeysalso{/tikz/intercept/@length = \pgfmathresult}
    \else
      \pgfkeysalso{/tikz/intercept/@length = #1}
    \fi
    \typeout{=========================}
    \typeout{/tikz/intercept/@length:\pgfkeysvalueof{/tikz/intercept/@length}}
    \typeout{=========================} 
  },
  % intercept={A,B}: intercepts a line segment(starting 
  % from point A) of a certain length on line AB.
  intercept/.style args = {#1,#2}{
    insert path = {
      let 
        \n1 = {\pgfkeysvalueof{/tikz/intercept/@length}},
        \n2 = {\pgfkeysvalueof{/tikz/intercept/scale}},
        \p{AB} = ($(#2)-(#1)$),
        \n{d} = {veclen(\p{AB})},
        \n3 = {scalar(\n1*\n2/\n{d})}
      in ($(#1)!\n3!(#2)$)
    }
  },
  % intersect={A,B,C,D}: returns the intersection coordinate 
  % of line AB and line CD. 
  % https://en.wikipedia.org/wiki/Line%E2%80%93line_intersection
  intersect/.style args = {#1,#2,#3,#4}{
    insert path = {
      let 
        \p1 = (#1), \p2 = (#2), \p3 = (#3), \p4 = (#4), 
        \n1 = {intersectll(\x1,\y1,\x2,\y2,\x3,\y3,\x4,\y4)},
      in ($(\p1)!\n1!(\p2)$)
    }
  },
  % ========== Triangle Centers ==========
  % calculated from barycentric coordinates
  % incenter = {A,B,C}
  incenter/.style args = {#1,#2,#3}{
    insert path = {
      let
        \p1 = (#1), \p2 = (#2), \p3 = (#3),
        \p{AB} = ($(#2)-(#1)$),
        \p{BC} = ($(#3)-(#2)$),
        \p{CA} = ($(#1)-(#3)$),
        \n{a} = {veclen(\x{BC}, \y{BC})},
        \n{b} = {veclen(\x{CA}, \y{CA})},
        \n{c} = {veclen(\x{AB}, \y{AB})},
        \n{s} = {\n{a}+\n{b}+\n{c}},
        \n1 = {\n{a}/\n{s}}, 
        \n2 = {\n{b}/\n{s}},
        \n3 = {\n{c}/\n{s}},
      in ({\n1*\x1+\n2*\x2+\n3*\x3,\n1*\y1+\n2*\y2+\n3*\y3})
    }
  },
  % excenter = {A,B,C}, returns excenter opposite to the vertex A
  excenter/.style args = {#1,#2,#3}{ 
    insert path = {
      let
        \p1 = (#1), \p2 = (#2), \p3 = (#3),
        \p{AB} = ($(#2)-(#1)$),
        \p{BC} = ($(#3)-(#2)$),
        \p{CA} = ($(#1)-(#3)$),
        \n{a} = {veclen(\x{BC}, \y{BC})},
        \n{b} = {veclen(\x{CA}, \y{CA})},
        \n{c} = {veclen(\x{AB}, \y{AB})},
        \n{s} = {-\n{a}+\n{b}+\n{c}},
        \n1 = {\n{a}/\n{s}}, 
        \n2 = {\n{b}/\n{s}},
        \n3 = {\n{c}/\n{s}},
      in ({-\n1*\x1+\n2*\x2+\n3*\x3,-\n1*\y1+\n2*\y2+\n3*\y3})
    }
  },
  % circumcenter = {A,B,C}
  circumcenter/.style args = {#1,#2,#3}{
    insert path = {
      let
        \p1 = (#1), \p2 = (#2), \p3 = (#3),
        \p{AB} = ($(#2)-(#1)$),
        \p{BC} = ($(#3)-(#2)$),
        \p{CA} = ($(#1)-(#3)$),
        \n{a} = {veclen(\x{BC}, \y{BC})},
        \n{b} = {veclen(\x{CA}, \y{CA})},
        \n{c} = {veclen(\x{AB}, \y{AB})},
        \n{m} = {max(max(\n{a},\n{b}),\n{c})},
        \n{a} = {\n{a}/\n{m}},
        \n{a} = {\n{a}*\n{a}},
        \n{b} = {\n{b}/\n{m}},
        \n{b} = {\n{b}*\n{b}},
        \n{c} = {\n{c}/\n{m}},
        \n{c} = {\n{c}*\n{c}},
        \n1 = {\n{a}*(\n{b}+\n{c}-\n{a})},
        \n2 = {\n{b}*(\n{c}+\n{a}-\n{b})},
        \n3 = {\n{c}*(\n{a}+\n{b}-\n{c})},
        \n{s} = {\n1+\n2+\n3},
        \n1 = {\n1/\n{s}},
        \n2 = {\n2/\n{s}},
        \n3 = {\n3/\n{s}},
      in ({\n1*\x1+\n2*\x2+\n3*\x3,\n1*\y1+\n2*\y2+\n3*\y3})
    }
  },
  % orthocenter = {A,B,C}
  orthocenter/.style args = {#1,#2,#3}{
    insert path = {
      let
        \p1 = (#1), \p2 = (#2), \p3 = (#3),
        \p{AB} = ($(#2)-(#1)$),
        \p{BC} = ($(#3)-(#2)$),
        \p{CA} = ($(#1)-(#3)$),
        \n{a} = {veclen(\x{BC}, \y{BC})},
        \n{b} = {veclen(\x{CA}, \y{CA})},
        \n{c} = {veclen(\x{AB}, \y{AB})},
        \n{m} = {max(max(\n{a},\n{b}),\n{c})},
        \n{a} = {\n{a}/\n{m}},
        \n{a} = {\n{a}*\n{a}},
        \n{b} = {\n{b}/\n{m}},
        \n{b} = {\n{b}*\n{b}},
        \n{c} = {\n{c}/\n{m}},
        \n{c} = {\n{c}*\n{c}},
        \n{a2} = {\n{b}+\n{c}-\n{a}},
        \n{b2} = {\n{c}+\n{a}-\n{b}},
        \n{c2} = {\n{a}+\n{b}-\n{c}},
        \n1 = {\n{c2}*\n{b2}},
        \n2 = {\n{a2}*\n{c2}},
        \n3 = {\n{b2}*\n{a2}},
        \n{s} = {\n1+\n2+\n3},
        \n1 = {\n1/\n{s}},
        \n2 = {\n2/\n{s}},
        \n3 = {\n3/\n{s}},
      in ({\n1*\x1+\n2*\x2+\n3*\x3,\n1*\y1+\n2*\y2+\n3*\y3})
    }
  },
  % centroid = {A,B,C}
  centroid/.style args = {#1,#2,#3}{
    insert path = {
      let
        \p1 = (#1), \p2 = (#2), \p3 = (#3),
      in ({(\x1+\x2+\x3)/3},{(\y1+\y2+\y3)/3})
    }
  },
  % nine-pint center = {A,B,C}
  nine-point center/.style args = {#1,#2,#3}{
    insert path = {
      let
        \p1 = (#1), \p2 = (#2), \p3 = (#3),
        \p{AB} = ($(#2)-(#1)$),
        \p{BC} = ($(#3)-(#2)$),
        \p{CA} = ($(#1)-(#3)$),
        \n{a} = {veclen(\x{BC}, \y{BC})},
        \n{b} = {veclen(\x{CA}, \y{CA})},
        \n{c} = {veclen(\x{AB}, \y{AB})},
        \n{m} = {max(max(\n{a},\n{b}),\n{c})},
        \n{a} = {\n{a}/\n{m}},
        \n{a} = {\n{a}*\n{a}},
        \n{b} = {\n{b}/\n{m}},
        \n{b} = {\n{b}*\n{b}},
        \n{c} = {\n{c}/\n{m}},
        \n{c} = {\n{c}*\n{c}},
        \n1 = {\n{a}*(\n{b}+\n{c})-(\n{b}-\n{c})*(\n{b}-\n{c})},
        \n2 = {\n{b}*(\n{c}+\n{a})-(\n{c}-\n{a})*(\n{c}-\n{a})},
        \n3 = {\n{c}*(\n{a}+\n{b})-(\n{a}-\n{b})*(\n{a}-\n{b})},
        \n{s} = {\n1+\n2+\n3},
        \n1 = {\n1/\n{s}},
        \n2 = {\n2/\n{s}},
        \n3 = {\n3/\n{s}},
      in ({\n1*\x1+\n2*\x2+\n3*\x3,\n1*\y1+\n2*\y2+\n3*\y3})
    }
  },
  % ========== Circle Operations ==========
  % circle = {O,A}, creates circle with the center (O) through A
  circle/.style args = {#1,#2}{
    insert path = {
      let
        \p{OA} = ($(#2)-(#1)$),
      in (#1) circle ({veclen(\p{OA})})
    }
  },
  % tagent point = {O,A,P}
  % O,A: center of circle and an abitary point on the circle 
  % P: a point outside the circle
  tangent point/.style args = {#1,#2,#3}{
    insert path = {
      let 
        \p{OA} = ($(#2)-(#1)$), % 半径
        \p{OP} = ($(#3)-(#1)$), 
        \n1 = {veclen(\p{OA})},
        \n2 = {veclen(\p{OP})},
        \n3 = {scalar(\n1/\n2)}
      in ($(#1)!\n3!{acos(\n1/\n2)}:(#3)$)
    }
  },
  % external homothetic center
  % O1,A1: center of circle 1 and an abitary point on the circle 
  % O2,A2: center of circle 2 and an abitary point on the circle
  external center/.style args = {#1,#2,#3,#4}{
    insert path = {
      let 
        \p{O1A1} = ($(#2)-(#1)$),% 半径O1A1
        \p{O2A2} = ($(#4)-(#3)$),% 半径O2A2
        \n{r1} = {veclen(\p{O1A1})},
        \n{r2} = {veclen(\p{O2A2})},
        \n1 = {scalar(\n{r1}/(\n{r1}-\n{r2}))}
      in ($(#1)!\n1!(#3)$)
    }
  },
  % internal homothetic center
  % O1,A1: center of circle 1 and an abitary point on the circle 
  % O2,A2: center of circle 2 and an abitary point on the circle
  internal center/.style args = {#1,#2,#3,#4}{
    insert path = {
      let 
        \p{O1A1} = ($(#2)-(#1)$),% 半径O1A1
        \p{O2A2} = ($(#4)-(#3)$),% 半径O2A2
        \n{r1} = {veclen(\p{O1A1})},
        \n{r2} = {veclen(\p{O2A2})},
        \n1 = {scalar(\n{r1}/(\n{r1}+\n{r2}))}
      in ($(#1)!\n1!(#3)$)
    }
  },
  % creates the radical axis of two non-concentric circles
  % O1,A1: center of circle 1 and an abitary point on the circle 
  % O2,A2: center of circle 2 and an abitary point on the circle
  radical axis/.style args = {#1,#2,#3,#4}{
    insert path = {
      let 
        \n{s} = {\pgfkeysvalueof{/tikz/start modifier}},
        \n{e} = {\pgfkeysvalueof{/tikz/end modifier}},
        \p{O1A1} = ($(#2)-(#1)$),% 半径O1A1
        \p{O2A2} = ($(#4)-(#3)$),% 半径O2A2
        \p{O1O2} = ($(#3)-(#1)$),
        \n{r1} = {veclen(\p{O1A1})},
        \n{r2} = {veclen(\p{O2A2})},
        \n{d} = {veclen(\p{O1O2})},
        \n1 = {scalar(\n{r1}/\n{d})},
        \n2 = {scalar(\n{r2}/\n{d})},
        \n3 = {.5*(1+\n1*\n1-\n2*\n2)},
        \p{ft} = ($(#1)!\n3!(#3)$),% perpendicular foot
        \p{s0} = ($(\p{ft})+(-\y{O1O2},\x{O1O2})$),
        \p{e0} = ($(\p{ft})+(\y{O1O2},-\x{O1O2})$),
        \p{s} = ($(\p{s0})!\n{s}!(\p{e0})$),% start   
        \p{e} = ($(\p{s0})!\n{e}!(\p{e0})$)% end    
      in (\p{s}) -- (\p{e})
    }
  },
  % ========== Path Definitions ==========
  % perpendicular bisector of the line segment (#1 -- #2)
  perpendicular bisector/.style args = {#1,#2}{
    insert path = { 
      let
        \n{s} = {\pgfkeysvalueof{/tikz/start modifier}},
        \n{e} = {\pgfkeysvalueof{/tikz/end modifier}},
        \p{AB} = ($(#2)-(#1)$),
        \p{m} = ($(#1)!0.5!(#2)$),% midpoint
        \p{s0} = ($(\p{m})+(-\y{AB},\x{AB})$),% rotate ccw, default start
        \p{e0} = ($(\p{m})+(\y{AB},-\x{AB})$),% rotate cw, default end
        \p{s} = ($(\p{s0})!\n{s}!(\p{e0})$),% start
        \p{e} = ($(\p{s0})!\n{e}!(\p{e0})$)% end
      in (\p{s}) -- (\p{e})
    }
  },
  % perpendicular line of the line (#1 -- #2) through #3
  % specifying start and end with modifiers(see tikz manual 13.5) 
  perpendicular/.style args = {#1,#2,#3}{
    insert path = {
      let
        \n{s} = {\pgfkeysvalueof{/tikz/start modifier}},
        \n{e} = {\pgfkeysvalueof{/tikz/end modifier}},
        \p{AB} = ($(#2)-(#1)$),
        \p{ft} = ($(#1)!(#3)!(#2)$),% perpedicular foot  
        \p{s0} = ($(\p{ft})+(-\y{AB},\x{AB})$),
        \p{e0} = ($(\p{ft})+(\y{AB},-\x{AB})$),
        \p{s} = ($(\p{s0})!\n{s}!(\p{e0})$),% start   
        \p{e} = ($(\p{s0})!\n{e}!(\p{e0})$)% end    
      in (\p{s}) -- (\p{e})
    }
  },
  % parallel line of the line (#1 -- #2) through #3
  % specifying start and end with modifiers(see tikz manual 13.5) 
  parallel/.style args = {#1,#2,#3}{
    insert path = {
      let
        \n{s} = {\pgfkeysvalueof{/tikz/start modifier}},
        \n{e} = {\pgfkeysvalueof{/tikz/end modifier}},
        \p{s0} = (#3),
        \p{e0} = ($(#3)+(#2)-(#1)$),    
        \p{s} = ($(\p{s0})!\n{s}!(\p{e0})$),% start   
        \p{e} = ($(\p{s0})!\n{e}!(\p{e0})$)% end  
      in (\p{s}) -- (\p{e})
    }
  },
  % alias for parallel={A,B,A}
  extend/.style args = {#1,#2} {
    parallel={#1,#2,#1}
  },
  % rotate around the origin by `angle` and then shift by (`xshift`,`yshift`)
  % tansform = {angle:(xshift,yshift)}
  transform/.style args = {#1:(#2,#3)} {
    cm={cos(#1), sin(#1), -sin(#1), cos(#1), (#2,#3)}
  },
}

% Utilities for implementation of 'intercept'
\def\euclidea@ComputeLength#1,#2\euclidea@stop{
  \newdimen\euclidea@ax
  \newdimen\euclidea@ay
  \newdimen\euclidea@bx
  \newdimen\euclidea@by
  \pgfextractx{\euclidea@ax}{\pgfpointanchor{#1}{center}}
  \pgfextracty{\euclidea@ay}{\pgfpointanchor{#1}{center}}
  \pgfextractx{\euclidea@bx}{\pgfpointanchor{#2}{center}}
  \pgfextracty{\euclidea@by}{\pgfpointanchor{#2}{center}}
  % 以下 showthe 指令 overleaf.com 编译通过, 而在 macOS+texlive 2021 报错
  % \showthe\euclidea@ax
  % \showthe\euclidea@ay
  % \showthe\euclidea@bx
  % \showthe\euclidea@by
  \pgfmathveclen{\euclidea@ax-\euclidea@bx}{\euclidea@ay-\euclidea@by}
}


% Syntax of parabola: y = a * x * x + b * x + c
%   \parabola [options] (a,b,c); 
\newcommand\parabola{} % just for safety
\def\parabola[#1]#2(#3,#4,#5){
  \path[smooth,domain=-2:2,#1,variable=\euclidea@var]
    plot({\euclidea@var},{(#3)*\euclidea@var*\euclidea@var+(#4)*\euclidea@var+(#5)})
}

% Syntax of ellipse:
%   \ellipse [options] (a,b); 
% wherein, a,b are major/minor semi axis.
\newcommand\ellipse{} % just for safety
\def\ellipse[#1]#2(#3,#4){
  \path[#1] (0,0) ellipse ({#3} and {#4})
}

% Syntax of hyperbola:
%   \hyperbola [options] (a,b); 
% wherein, a,b are major/minor semi axis.
\newcommand\hyperbola{} % just for safety
\def\hyperbola[#1]#2(#3,#4){
  \path[smooth,domain=-1.5:1.5,#1,variable=\euclidea@var]
    plot({-(#3)*cosh(\euclidea@var)},{(#4)*sinh(\euclidea@var)})  % right arm
    plot({ (#3)*cosh(\euclidea@var)},{(#4)*sinh(\euclidea@var)})  % left arm
}

% Syntax of asymptote:
%   \asymptote [options] (a,b); 
% wherein, a,b are major/minor semi axis.
\newcommand\asymptote{} % just for safety
\def\asymptote[#1]#2(#3,#4){
  \path[domain=-2:2,#1,variable=\euclidea@var,]
    plot({\euclidea@var},{ (#4)/(#3)*(\euclidea@var)})  
    plot({\euclidea@var},{-(#4)/(#3)*(\euclidea@var)}) 
}

% Syntax of axes:
%   \axes (xmin:xmax, ymin:ymax);
\newcommand\axes{}
\def\axes(#1:#2,#3:#4){
  \draw[help lines] (#1,#3) grid[step=1] (#2,#4);
  \draw[-latex] ({(#1)-0.5},0) -- ({(#2)+0.5},0) node[below] {$x$};
  \draw[-latex] (0,{(#3)-0.5}) -- (0,{(#4)+0.5}) node[left] {$y$};
  \draw (0,0) node[below left] {$O$}
}

\makeatother
